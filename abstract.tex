% % In the realm of photoemission spectroscopy, the exploration of large multi-dimensional phase spaces necessitates time-intensive data acquisition to ensure statistical robustness. Despite the unparalleled capabilities of free-electron lasers (FELs), in peak brightness and ultra- short pulsed X-rays, the limitations of low repetition rates prolong the data acquisition process. This impedes the agility of decision making that could otherwise enhance experimental results in the limited and valuable beamtime. By employing denoising strategies to mitigate noise while preserving intrinsic information, our proposed approach aims to streamline the data acquisition process, and effectively manage the escalating size and complexity of multi-dimensional photoemission data.
% We present a deep learning approach based on the Noise2Noise framework to denoise multidimensional photoemission spectroscopy (MPES) data obtained with a time-of-flight momentum microscope. Specifically, a 3D U-Net architecture is trained using low- and high-count noisy data, enabling the model to learn noise characteristics without requiring clean images. Our approach excels at reconstructing images even at extremely low count levels (order of 10^-3 counts/pixel), where conventional denoising techniques simply fail. Tests show that a 10-min acquisition processed with our deep learning model resolves major features not even visible after multiple hours of measurement. The presented approach has the potential to streamline the MPES data acquisition process at table-top/laboratory sources as well as large-scale facilities like FEL FLASH. By utilizing our method in future studies, researchers will be able to efficiently optimize acquisition parameters; thus, significant beamtime could be conserved, or an existing beamtime budget could be used more effectively, allowing for the exploration of a broader parameter space.