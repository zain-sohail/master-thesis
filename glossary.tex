\newglossaryentry{train}{
    name={\textit{train}},
    description={Also known as macrobunch. A train represents a group of closely spaced electron bunches produced and accelerated by the \gls{FEL} (or more generally, any accelerator). Each train is associated with a unique identifier called \texttt{trainId}, which is used as the primary index for much of the data reduction process.}
}

\newglossaryentry{pulse}{
    name={\textit{pulse}},
    description={Also known as \gls{microbunch}. Each \gls{train} contains about 500 pulses produced from the \gls{SASE} process (see \gls{sase}). These are which are used as a secondary index for the data reduction process.},
    plural={\textit{pulses}},
}

\newglossaryentry{sase}{
    name={\textit{self-amplified spontaneous emission}},
    description={is a process where the electron beam in the accelerator, when passing through an undulator, starts emitting radiation due to acceleration. The interaction between the emitted radiation and the charge distribution leads to microbunching. These microbunches emit radiation coherently, leading to the intense, coherent radition, characteristic of an \gls{FEL}.} %For more information see \cite{ackermannOperationFreeelectronLaser2007}}
}

\newglossaryentry{microbunch}{
    name={\textit{microbunching}},
    description={Microbunches are produced by the interaction between the oscillating electrons in the undulator and the radiation that they produce (due to the oscillatory acceleration) leads to periodic longitudinal density modulation known as Microbunching. The in-phase emitted radiation adds coherently, increasing intensity and enhancing microbunching.},
    plural={Microbunches},
}

\newglossaryentry{spontaneous_emission}{
    name={\textit{spontaneous emission}},
    description={Spontaneous emission requires no external perturbation, and is explained in quantum electrodynamics by the interaction between an atom and quantized electromagnetic field, where even with no photons in the field, there is a non-zero probability of photon emission from the atom.}
}

% \newglossaryentry{stimulated_emission}{
%     name={Stimulated Emission},
%     description={}
% }

% \newglossaryentry{fel}{
%     name={free-electron laser},
%     description={
%     is an x-ray radiation source; fundamentally comprising of a linear particle accelerator and an \gls{undulator} (or a series of undulators). The accelerator produces a bunched electron beam similar to that of a synchrotron, which can be compressed to reach ultrashort pulse duration (femtosecond) with peak brightness many orders of magnitude above synchrotrons. Since the electrons move in a vacuum, it is termed Free-Electron in comparison to traditional lasers which are bound by the materials energy levels. Whereas, it is called a Laser due to there being light amplification and the shared properties with traditional optical lasers such as high pulse energy and being coherent. Taken from \cite{sohailUltrafastDynamicStudies2021}}
% }

\newglossaryentry{undulator}{
    name={\textit{undulator}},
    description={Magnets arranged periodically to produce a periodic magnetic field. It is used to produce coherent radiation by accelerating electrons through it}
}

\newglossaryentry{beamline}{
    name={\textit{beamline}},
    description={A path leading the photons from the particle accelerator to the experimental end-station.}
}

\newglossaryentry{beamtime}{
    name={\textit{beamtime}},
    description={The time allocated to an experiment at a synchrotron or \gls{FEL} facility.}
}

% shot noise
\newglossaryentry{shot_noise}{
    name={\textit{shot noise}},
    description={Shot noise originates from the quantized nature of light. It is the noise that arises from the random arrival of photons at a detector.}
}

% gas monitor detector
\newglossaryentry{gmd}{
    name={\textit{gas monitor detector}},
    description={A diagnostic tool to measure the intensity of a \gls{FEL} pulse in a non-invasive manner. The gas inside the detector is ionized by the \gls{FEL} pulse, and the resulting current can be used to detect the absolute number of photons with an accuracy of 10\%.}
}

% latent
\newglossaryentry{latent}{
    name={\textit{latent}},
    description={Latent refers to the underlying distribution of the data, which is unobserved.}
}

% noise
\newglossaryentry{noise}{
    name={\textit{noise}},
    description={The inherent fluctuations in data due to its stochastic nature.}
}
%noise2noise
\newglossaryentry{noise2noise}{
    name={\textit{Noise2Noise}},
    description={A training paradigm where both the input and target datasets are noisy, eliminating the need for clean reference data.}
}

%space-charge effect
\newglossaryentry{space-charge}{
    name={\textit{space-charge effect}},
    description={The space-charge effect occurs when intense light pulses generates a large electron density. This leads to the Coulomb repulsion between electrons, causing a distortion in the electric field, leading to a spread in the energy of the electrons}
}

%empirical risk/training error
\newglossaryentry{training_error}{
    name={\textit{training error}},
    description={Also known as \textit{empirical risk}. The error between the predicted output and the target output on the training dataset measured using a loss function.}
}

%generalization error
\newglossaryentry{generalization}{
    name={\textit{generalization}},
    description={The ability of a learning based model to perform well on unseen data.}
}
