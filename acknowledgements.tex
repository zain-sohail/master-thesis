% Thank Dima especially
% Dataset providers
% Thank Lorenz Kruger, Martin and others in group (Lukas Weigand).
\dots

This research was supported in part through the Maxwell computational resources operated at \gls{DESY}, Hamburg, Germany

The \texttt{Python} data science and visualization ecosystem was heavily employed in this thesis.
The usage of \texttt{Xarray} \cite{hoyerXarrayNDLabeled2017} for multidimensional dataset transformations, \texttt{matplotlib} \cite{hunterMatplotlib2DGraphics2007} for plotting of images and other figures, \texttt{pandas} \cite{thepandasdevelopmentteamPandasdevPandasPandas2024} for tabular assessment of data, \texttt{seaborn} \cite{waskomSeabornStatisticalData2021} for statistical plotting, \texttt{optuna} \cite{akibaOptunaNextgenerationHyperparameter2019} for hyperparameter optimization, are acknowledged by citation as this is preferred by these scientific libraries. 



Moreover, HDF5, \texttt{SED} as it allows to easily work with \gls{MPES} data, and many other packages.

For deep learning, exclusively \texttt{PyTorch} \cite{paszkePyTorchImperativeStyle2019} has been used. We use the \texttt{UNET3D} shown in \cite{cicek3DUNetLearning2016}, implemented (modified to accommodate our needs) by \citeauthor{wolnyAccurateVersatile3D2020} \cite{wolnyAccurateVersatile3D2020}.


\dots

Many of the concepts about Statistical Learning Theory were introduced to the author by the lecture series \texttt{Algorithmic Foundations of Data Science} taught by Prof. Martin Grohe at the RWTH Aachen University.
