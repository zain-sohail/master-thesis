This thesis would not have been possible without the constant support and guidance of the many people that supported me through the process. Though extremely challenging in the breath and depth I tried to uptake, it allowed me to explore many directions of research and allowed me to learn a lot, and for that I am thankful.

I would like to especially express my gratitude to my supervisors B. Berkels and Kai, and Dima, my acting supervisor throughout the thesis. I am very thankful to B. Berkels for accepting to supervise this external research project. I appreciate the commitment made to constantly guide me in the right direction, especially in the fields of image denoising and mathematics relating. I am thankful to Kai for the detailed discussions on exploring different physics, and supporting me in presenting my research. I would especially like to thank Dima for the consistent guidance throughout this project, for helping me understand all the details surrounding the experimental aspects of MPES, and for always being very understanding.

Naturally, no work is complete without the help of those close and dear. I extend my heartfelt thanks to the closest for the consistent support through the fun and the tough times. Shoutout to Kumarah who stayed up late nights helping me with corrections, in a domain he had little clue of.

I also extend my gratitude to Martin Burger's group. Especially to Lorenz Kruger for the insightful conversations trying to explore the data, its statistical implications and more.

This contribution was made possible through the generosity of the authors who shared their datasets: groups from DESY/FLASH, University of Mainz, FU Berlin, ETH Zürich for the \gls{GdW} dataset \cite{kutnyakhovMultidimensionalPhotoemissionSpectra2024}, \citeauthor{heberMultispectralTimeresolvedEnergy2022} and groups from ETH Zürich, DESY/NanoLab and Aarhus University for the \gls{GrIr} dataset \cite{heberMultispectralTimeresolvedEnergy2022}, and \citeauthor{maklarTimeresolvedARPESRAW2022} for the \gls{WSe2} dataset \cite{maklarTimeresolvedARPESRAW2022}.

We acknowledge DESY (Hamburg, Germany), a member of the Helmholtz Association HGF, for the provision of experimental facilities. Parts of this research were carried out at PG2 beamline of FLASH. This research was also supported in part through the Maxwell computational resources operated at DESY, Hamburg, Germany.

The \texttt{Python} data science and visualization ecosystem was heavily employed in this thesis.
The usage of \texttt{Xarray} \cite{hoyerXarrayNDLabeled2017} for multidimensional dataset transformations, \texttt{matplotlib} \cite{hunterMatplotlib2DGraphics2007} for plotting of images and other figures, \texttt{pandas} \cite{thepandasdevelopmentteamPandasdevPandasPandas2024} for tabular assessment of data, \texttt{seaborn} \cite{waskomSeabornStatisticalData2021} for statistical plotting, \texttt{optuna} \cite{akibaOptunaNextgenerationHyperparameter2019} for hyperparameter optimization, are acknowledged by citation as this is preferred by these scientific libraries. 

Extensive use of the \texttt{SED} (\href{https://github.com/OpenCOMPES/sed}{https://github.com/OpenCOMPES/sed}) is made, especially allowing easy manipulation of the single-event dataframes with $>$\num{1e9} rows, extremely fast binning to multidimensional images and compatibility with HDF5 and tiff formats.

For deep learning, exclusively \texttt{PyTorch} \cite{paszkePyTorchImperativeStyle2019} has been used. We use the \texttt{UNET3D} shown in \cite{cicek3DUNetLearning2016}, implemented (modified to accommodate our needs) by \citeauthor{wolnyAccurateVersatile3D2020} \cite{wolnyAccurateVersatile3D2020}. \texttt{PyTorch} \texttt{Dataset} and \texttt{DataLoader} classes were extensively used to ease in experiments other than deep learning.

Many of the concepts about Statistical Learning Theory were introduced to the author by the lecture series \textit{Algorithmic Foundations of Data Science} taught by Prof. M Grohe at the RWTH Aachen University. And much of their mathematical understanding on signal and image processing is based on the lecture series \textit{Mathematical Methods of Signal and Image processing} taught by Prof. B Berkels.
