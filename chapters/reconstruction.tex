Rather than calling it denoising, better word in image reconstruction because
Image Reconstruction:

Purpose: To reconstruct an image from incomplete, noisy, or indirect measurements. This is often used in medical imaging (e.g., MRI, CT scans), computational photography, and computer vision applications. 

Reconstruction involves generating a complete image from partial or indirect data, which can include denoising and deblurring as sub-tasks.

\section{Address reconstruction/denoising schemes}
VST with BM3D: BM3D uses collaborative filtering, which is also used in recommender systems [citation need]
UNET noise2noise

Why ARPES transfer learning can't work for our case.


Most methods assume a Gaussian noise model, whether in classical or DL approaches. So we need to see the noise type first.

"Owing to solve the clean image x from the Eq. (1) is an ill-posed problem, we cannot get the unique solution from the image model with noise. To obtain a good estimation image 
, image denoising has been well-studied in the field of image processing over the past several years. Generally, image denoising methods can be roughly classified as [3]: spatial domain methods, transform domain methods, which are introduced in more detail in the next couple of sections." from \href{https://vciba.springeropen.com/articles/10.1186/s42492-019-0016-7}{source}

\section{Metrics}
\gls{PSNR}, \gls{SSIM}, \gls{MSE}, \gls{MAE}, Huber loss, Poisson loss are examples of metrics used to evaluate the quality of the reconstructed image. These metrics measure the similarity between the true image and the reconstructed image, and can be used to compare different reconstruction algorithms.

PSNR: Peak Signal-to-Noise Ratio can be written as:
\begin{equation}
    \text{PSNR} = 10 \log_{10} \left( \frac{255^2}{\text{MSE}} \right)
\end{equation}

SSIM: Structural Similarity Index Measure can be written as:
\begin{equation}
    \text{SSIM} = \frac{(2\mu_x\mu_y + C_1)(2\sigma_{xy} + C_2)}{(\mu_x^2 + \mu_y^2 + C_1)(\sigma_x^2 + \sigma_y^2 + C_2)}
\end{equation}

\section{BM3D: Densoising in sparse domain}
with and without anscombe on different datasets from flash, lab and fhi
Testing s
\subsection{Anscombe: Variance Stabilization Transform}
